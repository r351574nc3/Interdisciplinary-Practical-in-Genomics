\documentclass[12pt,notitlepage]{article}
\author{Leo Przybylski}
\usepackage{graphicx}
\usepackage{listings}
\usepackage{color}
\usepackage{hyperref}
\usepackage{hyperlatex}
\setcounter{htmldepth}{0}

\definecolor{DarkBlue}{rgb}{0,0,0.55}
\definecolor{DarkGreen}{rgb}{0,0.4,0}
\definecolor{Purple}{rgb}{0.5,0,0.5}

\begin{ifhtml}
\newcommand{\HlxSubTitle}{}
\newcommand{\HlxSubTitleP}{}
\newcommand{\subtitle}[1]{\renewcommand{\HlxSubTitleP}{1}%
  \renewcommand{\HlxSubTitle}{#1}}
\newcommand{\HlxAffiliation}{}
\newcommand{\HlxAffiliationP}{}
\newcommand{\affiliation}[1]{\renewcommand{\HlxAffiliationP}{1}%
  \renewcommand{\HlxAffiliation}{#1}}
\newcommand{\sf}[1]{\xml{span style="font-family: sans-serif;"}#1\xml{/span}}
\newcommand{\rm}[1]{\xml{span style="font-family: serif;"}#1\xml{/span}}
\newcommand{\bf}[1]{\xml{span style="font-weight: bold;"}#1\xml{/span}}
\newcommand{\href}[2]{\xml{a href="#1"}#2\xml{/a}}
\newcommand{\HlxStyleSheet}{
  \begin{rawxml}
    <!-- metadata -->
    <meta name="generator" content="S5" />
    <meta name="version" content="S5 1.1" />
    <meta name="company" content="Leosandbox " />
    <!-- configuration parameters -->
    <meta name="defaultView" content="slideshow" />
    <meta name="controlVis" content="hidden" />
    <!-- style sheet links -->
    <link rel="stylesheet" href="ui/kuali/slides.css" type="text/css" media="projection" id="slideProj" />
    <link rel="stylesheet" href="ui/kuali/outline.css" type="text/css" media="screen" id="outlineStyle" />
    <link rel="stylesheet" href="ui/kuali/print.css" type="text/css" media="print" id="slidePrint" />
    <link rel="stylesheet" href="ui/kuali/opera.css" type="text/css" media="projection" id="operaFix" />
    <script src="ui/kuali/slides.js" type="text/javascript"></script>

  \end{rawxml}
}

\newcommand{\maketitle}{
    \xml{div class="slide"}
    \xml{h1}\HlxTitle\xml{/h1}
    \xml{h2}\HlxSubTitle\xml{/h2}
    \xml{h3}\HlxAuthor\xml{/h3}
    \xml{h4}\HlxAffiliation\xml{/h4}
    \xml{/div}}
\end{ifhtml}

\newenvironment{s5presentation}{\endpar%
  \HlxBlk\begin{rawxml}
  <div class="header">
    <div class="header_l">
      <div class="header_r">
      &nbsp;
      </div>
    </div>
  </div>
  <div class="content">
    <div class="content_l">
      <div class="content_r">
        <img src="ui/kuali/blank.gif" id="filler" border="0"/>
<div class="layout">
  <div id="controls"><!-- DO NOT EDIT --></div>
  <div id="currentSlide"><!-- DO NOT EDIT --></div>
  <div id="header">
    <img style="position: relative; top:30px; left: 18px;" id="logo" src="ui/kuali/logo.png" />
  </div>
  <div id="footer">
    <div class="footer_l">
      <div class="footer_r">
        <h1/>
        <h2>\end{rawxml}\HlxDate\begin{rawxml}</h2>
      </div>
    </div>
  </div>
</div>
    </div>
  </div>
 </div>
<div class="presentation">
    \end{rawxml}}{\HlxBlk\xml{/div}}
\newenvironment{s5slide}{\xml{div class="slide"}
\newenvironment{s5notes}{\xml{div class="notes"}}{\xml{/div}}
}{\xml{/div}}
\newcommand{\fulltitle}[1]{\title{#1}\htmlonly{\htmltitle{#1}}}
\newcommand{\thumbnail}[2]{\endpar%
\xml{a href="#1" id="external" rel="lightbox[gallery]"}\xml{img src="#2" id="external" border="0" width="100"/}\xml{/a}
}
\newenvironment{slideshow}{\endpar%
  \HlxBlk\xml{ul class="gallery"}\begingroup\stepcounter{listcounter}%
  \newcounter{gallerycounter}%
  \newcommand{\item}{%
      \setcounter{gallerycounter}{\thelistcounter}%
      \HlxBlk\xml{li}}\ignorespaces}%
    {\begin{ifequal}{\thegallerycounter}{\thelistcounter}%
        \xml{/li}\end{ifequal}\endgroup%
    \HlxBlk\xml{/ul}}
\fulltitle{iPIG Markov Model Implementation}
\W \subtitle{Analysis of Probability of Words in Nucleutide Sequences}

\W \affiliation{University of Arizona}


\begin{document}
  \W \begin{s5presentation}
  \maketitle

\begin{ifhtml}
    \begin{s5slide}
      \section{Overview}
      \begin{itemize}
      \item High-level view of entire program
      \item Execution and Input Parameters
      \item Reading Sequences
      \item Generating Words
      \item Calculating Occurrences
      \item Testing
      \item References
      \end{itemize}
    \end{s5slide}

    \begin{s5slide}
      \section{High-level View}
      \xml{img src="Diagrams/probable_words_Activity_sized.png" id="external" border="0" /}
    \end{s5slide}

    \begin{s5slide}
      \section{Execution and Input Parameters}
      \subsection{Check for probability in NM-MC58 nucleutide as a non-coding sequence with debug level DEBUG}
      \xml{tt}cat ../NM-MC58.gbk | perl probable_words.pl -n --debug=1\xml{/tt}
      \subsection{Check for probability in NM-MC58 nucleutide with debug level WARN. Check both the DUS and the reverse DUS.}
      \xml{tt}cat ../NM-MC58.gbk | perl probable_words.pl -r --debug=3\xml{/tt}
      \subsection{View the perldoc}
      \xml{tt}perl probable_words.pl --man\xml{/tt}
    \end{s5slide}

    \begin{s5slide}
      \section{Execution and Input Parameters}
     
      \begin{description}
        \item[dus] the DNA Uptake Sequence to use
        \item[debug] the debugging level to use
        \item[non-coding] use a non-coding or standard sequence?
        \item[order] order to use from 2-5
        \item[help] display command line help
        \item[man] show the man page
      \end{description}
    \end{s5slide}

    \begin{s5slide}
      \section{Reading Sequences}
      \xml{img src="Diagrams/read_sequences_Sequence_sized.png" id="external" border="0" /}
    \end{s5slide}

    \begin{s5slide}
      \section{Generating Words}
    \end{s5slide}

    \begin{s5slide}
      \section{Calculating Occurrences}
      \xml{img src="Diagrams/calculate_occurrences_Sequence_sized.png" id="external" border="0" /}
    \end{s5slide}

    \begin{s5slide}
      \section{Testing}
      \subsection{Scenarios}
      \begin{itemize}
      \item Line-separated 2-mers
      \item Line-separated 4-mers
      \item Line-separated 5-mers
      \item Invalid Characters
      \item Masked Sequences
      \end{itemize}
    \end{s5slide}

    \begin{s5slide}
      \section{Line-separated 2-mers}
    \end{s5slide}

    \begin{s5slide}
      \section{Line-separated 4-mers}
    \end{s5slide}

    \begin{s5slide}
      \section{Line-separated 5-mers}
    \end{s5slide}

    \begin{s5slide}
      \section{Invalid Characters}
    \end{s5slide}

    \begin{s5slide}
      \section{Masked Sequences}
    \end{s5slide}

    \begin{s5slide}
      \section{References}
      \begin{itemize}
      \item \href{http://github.com/r351574nc3/Interdisciplinary-Practical-in-Genomics}{iPIG GitHub Source Code Repository}
      \item \href{http://www.u.arizona.edu/~przybyls/probable_words.html}{probable_words.pl API}
      \item Thanks to Lauren Meyers for test data and content.
      \end{itemize}
    \end{s5slide}

\end{ifhtml}  

  \W \end{s5presentation}
\end{document}
