%% \tableofcontents

\section{Class \texttt{ClusterGraph}\label{Class_ClusterGraph}\index{Class ClusterGraph}}
\subsection*{Description\label{Description}\index{Description}}


Used to contain clusters that are intended to be sorted. Clusters
are sorted according to the cardinality of genes per cluster. For 
example, clusters with 12 genes will be grouped in an array together.
The cardinality of that array determines the order in which these
clusters appear. Since the matrix is sorted in ascending order,
group 12 will appear after 10 and 11 due to its cluster cardinality.
Such information is used later to be depicted in a visual graph.



Right now, a multidimensional array is used to do this, but it might
be easier to use an insertion sort and apply the Observer pattern.
This would require more classes, and I'm not sure I care that much.



An index is kept (a \texttt{HASH}) to key the index (the indices of the cluster
in the array storage) by a Gene. EVERY gene will points to an index
for a cluster.

\subsubsection*{Author: \textit{Leo Przybylski (przybyls@arizona.edu)}\label{Author:_Leo_Przybylski_przybyls_arizona_edu_}\index{Author: Leo Przybylski (przybyls@arizona.edu)}}
\subsection*{Default Constructor\label{Default_Constructor}\index{Default Constructor}}


Constructs the \texttt{ClusterGraph} from its attributes. None are required though. 
Initiates an array as its data store and saves a reference to it.

\subsubsection*{Parameters\label{Parameters}\index{Parameters}}
\begin{description}

\item[{\texttt{identity} - \% Identity lower bound}] \textbf{used for validation}
\item[{\texttt{alignment} - Alignment length lower}] \textbf{bound used for validation}\end{description}
\subsection*{Method \texttt{add}\label{Method_add}\index{Method add}}


Adds a \texttt{Cluster} to the graph. Sets the index to the last unoccupied index
and records it in the index \texttt{HASH}. Finally, the cluster is appended to the
array store.



This is the only place where the size of the \texttt{ClusterGraph} is incremented.

\subsubsection*{Parameters\label{Parameters}\index{Parameters}}
\begin{description}

\item[{\texttt{toadd} - Cluster to add}] \mbox{}\end{description}
\subsection*{Method \texttt{addGene}\label{Method_addGene}\index{Method addGene}}


Add an \texttt{Gene} to a \texttt{Cluster} in the \texttt{ClusterGraph}. First, try to locate a
\texttt{Cluster} with the same query id as the \texttt{Gene} to add. Make sure that this
cluster is also referenced by the subject/hit id. If one cannot be found,
then instantiate one.

\subsubsection*{Parameters\label{Parameters}\index{Parameters}}
\begin{description}

\item[{\texttt{toadd} - \texttt{Gene} instance}] \textbf{to add}\end{description}
\subsection*{Getter \texttt{clusters}\label{Getter_clusters}\index{Getter clusters}}


Getter for the the stored array of clusters. \texttt{clusters} is a read-only attribute.

\subsubsection*{Returns\label{Returns}\index{Returns}}


Gets the reference to \texttt{clusters} array.

\subsection*{Getter \texttt{cluster}\label{Getter_cluster}\index{Getter cluster}}


Getter for the the a specific \texttt{cluster} by its index. \texttt{cluster} is a read-only attribute.

\subsubsection*{Returns\label{Returns}\index{Returns}}


Gets the reference to \texttt{clusters} array.

\subsection*{Getter/Setter \texttt{identity}\label{Getter_Setter_identity}\index{Getter/Setter identity}}


Getter for the the \% identity lower bound.

\subsubsection*{Parameters\label{Parameters}\index{Parameters}}
\begin{description}

\item[{\texttt{identity} - \% Identity lower bound}] \textbf{used for validation}\end{description}
\subsubsection*{Returns\label{Returns}\index{Returns}}


Gets the \% identity lower bound

\subsection*{Getter/Setter \texttt{alignment}\label{Getter_Setter_alignment}\index{Getter/Setter alignment}}


Getter for the the \% alignment lower bound.

\subsubsection*{Parameters\label{Parameters}\index{Parameters}}
\begin{description}

\item[{\texttt{alignment} - Alignment length lower}] \textbf{bound used for validation}\end{description}
\subsubsection*{Returns\label{Returns}\index{Returns}}


Gets the alignment length lower bound

\subsection*{Getter \texttt{size}\label{Getter_size}\index{Getter size}}
\subsubsection*{Returns\label{Returns}\index{Returns}}


The number of \texttt{Cluster} instances that are part of this \texttt{ClusterGraph}

\subsection*{Method \texttt{graph}\label{Method_graph}\index{Method graph}}


Creates a data structure that can be used by the \texttt{GD::Graph} module.

\subsubsection*{Returns\label{Returns}\index{Returns}}


A sorted 2-dimensional array of the data to be represented in a visual graph
