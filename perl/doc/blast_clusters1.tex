\documentclass[11pt,notitlepage]{article}
\usepackage{graphicx}
\usepackage{amsmath}            % adds more math symbols
\usepackage{amssymb}
\usepackage{multicol}
\title{Blast Cluster Analysis Tool}
\author{Leo Przybylski\\
\texttt{przybyls@arizona.edu}}

\newcommand{\question}[2]{\textbf{#1.} #2}
\newcommand{\subquestion}[2]{\par\hspace{0.5cm} \textbf{#1)} #2}
\newenvironment{answer}{\endpar%

}

    % Give wider margins; gives more text per page.

\setlength{\topmargin}{0.00in}
\setlength{\textheight}{8.75in}
\setlength{\textwidth}{6.625in}
\setlength{\oddsidemargin}{0.0in}
\setlength{\evensidemargin}{0.0in}

\setlength{\parindent}{0.0cm}	% Don't indent the paragraphs
%\setlength{\parskip}{0.4cm}	% distance between paragraphs

\begin{document}
  \maketitle
  \tableofcontents

  \abstract{\ldots
}
  {\setlength{\baselineskip}%
           {0.0\baselineskip}
  \section*{\hfill Background}
  \hrulefill \par}
  When a blast query is made, results are recorded with the identifier of the query
  as well as the identifier for the result or hit. The convention used to name the
  each query is a combination of gene and contig number. The records of blast output
  can be organized into several different types of meaningful information.

  One way in particular is to group records related by query and or hit information as
  clusters. We can then provide statistical information on clusters. By grouping clusters
  with the same cardinality of edges, we can analyze trends between different families
  of bacteria.  

  {\setlength{\baselineskip}%
           {0.0\baselineskip}
  \section*{\hfill Clusters}
  \hrulefill \par}
  A Cluster is a set of edges where the query identifiers match. For example, in the
  following blast results:

  \noindent \verb|AP206_contig00001_4923-1612	cinerea_contig00013_10696-7277	47.65	1129	517	23	29	1103	31	1139	0.0	 884|

  \noindent \verb|AP206_contig00001_4923-1612	elongata_contig01464_47682-51131	46.11	1156	545	24	7	1103	13	1149	0.0	 851|
  \\

  The records belong in the same cluster because they were results of the same query. Not
  all records that match the query identifier become edges. The edge must also


  Originally,
  it was assumed that a cluster would only have edges that match the query; however, since
  the clusters are considered to be a set, it is possible to create a union between clusters 
  resulting in a new cluster containing edges from the two that were unioned.

  Unions between clusters is determined by whether two clusters have edges that relate. For example,
  if two edges share the same subject/hit identifier, then they are considered to be adjacent edges
  even though they are not in the same cluster. This means, that a union can be made between
  these two clusters.

\end{document}

